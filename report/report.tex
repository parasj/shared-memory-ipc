\documentclass[paper=a4, fontsize=11pt]{scrartcl} % A4 paper and 11pt font size

\usepackage[T1]{fontenc} % Use 8-bit encoding that has 256 glyphs
\usepackage{fourier} % Use the Adobe Utopia font for the document - comment this line to return to the LaTeX default
\usepackage{graphicx} % For pictures
\usepackage{booktabs} % For tables
\usepackage{tabularx} % For tables
\usepackage[english]{babel} % English language/hyphenation
\usepackage{amsmath,amsfonts,amsthm} % Math packages

\usepackage{lipsum} % Used for inserting dummy 'Lorem ipsum' text into the template

\usepackage{sectsty} % Allows customizing section commands
\allsectionsfont{\centering \normalfont\scshape} % Make all sections centered, the default font and small caps

\usepackage{fancyhdr} % Custom headers and footers
\pagestyle{fancyplain} % Makes all pages in the document conform to the custom headers and footers
\fancyhead{} % No page header - if you want one, create it in the same way as the footers below
\fancyfoot[L]{} % Empty left footer
\fancyfoot[C]{} % Empty center footer
\fancyfoot[R]{\thepage} % Page numbering for right footer
\renewcommand{\headrulewidth}{0pt} % Remove header underlines
\renewcommand{\footrulewidth}{0pt} % Remove footer underlines
\setlength{\headheight}{13.6pt} % Customize the height of the header

\numberwithin{equation}{section} % Number equations within sections (i.e. 1.1, 1.2, 2.1, 2.2 instead of 1, 2, 3, 4)
\numberwithin{figure}{section} % Number figures within sections (i.e. 1.1, 1.2, 2.1, 2.2 instead of 1, 2, 3, 4)
\numberwithin{table}{section} % Number tables within sections (i.e. 1.1, 1.2, 2.1, 2.2 instead of 1, 2, 3, 4)

% \setlength\parindent{0pt} % Removes all indentation from paragraphs - comment this line for an assignment with lots of text

%----------------------------------------------------------------------------------------
%	TITLE SECTION
%----------------------------------------------------------------------------------------

\newcommand{\horrule}[1]{\rule{\linewidth}{#1}} % Create horizontal rule command with 1 argument of height

\title{	
\normalfont \normalsize 
\textsc{Georgia Institute of Technology} \\ [25pt] % Your university, school and/or department name(s)
\horrule{0.5pt} \\[0.4cm] % Thin top horizontal rule
\huge CS 6210 Project 2 Report: Shared Memory IPC \\ % The assignment title
\horrule{2pt} \\[0.5cm] % Thick bottom horizontal rule
}

\author{Manas George, Paras Jain} % Your name

\date{\normalsize\today} % Today's date or a custom date

\begin{document}

\maketitle % Print the title

\section{TinyFile Architecture}
The TinyFile application is structured as a client-server program. The server is a standalone daemon that accepts requests to compress files from clients. The client is a library that is linked into client applications that need to use the application. Client applications call exported functions from the client library to send data they want compressed to the compression server running separately as a daemon, which responds with the compressed version of the file.
\subsection{Server Design}
\subsubsection{Data Structures}
The server maintains state for each client it is currently communicating with, and calls out to the snappy-c library to actually perform the requested compression. Client state is tracked in a linked-list structure that records, for each client, the private message queue and the shared memory segment used to perform client-server communication.

% Insert linked list of structs  diagram

Apart from private message queues for each client, the server also has a global shared message queue that is available to all clients. This queue is used to initialize clients, as described in the following section, and for nothing else.

\subsubsection{Client Initialization}
When a new client comes online, it must register with the server using the \texttt{tiny\_initialize} function exported by the library. The initialization function uses the global shared message queue to send an initialization request to the server, invoking a series of initialization procedures on the server side that comprise the initial client-server handshake.
\begin{enumerate}
\item Memory is allocated to store the new client's state.
\item The client is assigned new unique id and a temporary file.
\item The temporary file is used to derive IPC keys.
\item The IPC key is used to initialize a private message queue and shared memory.
\item The new client is placed at the head of the global clients list.
\end{enumerate}

% Insert figure showing message passing for the client handshake.

Once all of this is done, the server responds with a message that contains the \texttt{client\_key} created from the client's temporary file, which is the only information the client needs to connect to its private message queue and shared memory segment. The initialization request and response are sent over the global shared message queue common to all clients. This introduces the possibility of a race condition in since there is no guarantee that a particular response will reach exactly the client it was sent from. This is not really a problem, as the clients are essentially indistinguishable until the handshake is completed, and so a situation in which a client reads an initialization response message intended for another client is never problematic; the original client will simply receive a response message intended for yet another client, and all clients end up with unique IDs and keys in the end. Notice that mixing up of messages is not an issue once the handshake is done. The global shared message queue is only used for initialization; all subsequent messages between a server and client go through a per-client client-private message queue that is initialized during the handshake.

% Insert figure highlighting the race condition.

\subsubsection{Compressing/Uncompressing Data}

\subsubsection{Cleaning Up}


\section{API}
\section{Performance}

\end{document}